\documentclass{article}

\usepackage{graphicx}
\usepackage{xeCJK}
\usepackage{bm}
\usepackage{amsmath,amsthm,amssymb,amsfonts}
\usepackage{cite}
\usepackage[colorlinks,linkcolor=red,anchorcolor=blue,citecolor=green,CJKbookmarks=true]{hyperref}
\usepackage{indentfirst}
\usepackage{amsmath}
\usepackage[margin=3.5cm]{geometry}
\usepackage{titlesec}
\usepackage{amsmath}
\usepackage{amssymb}
% \linespread{1.6}
\geometry{left=3.2cm,right=3.2cm,top=3.2cm,bottom=3.2cm}
\usepackage{multirow}
\usepackage{listings}
\usepackage{xcolor}
\usepackage{ulem}
\usepackage{enumitem}
\usepackage{tikz}
\usepackage{lipsum}
\setenumerate[1]{itemsep=0pt,partopsep=0pt,parsep=\parskip,topsep=5pt}
\setitemize[1]{itemsep=0pt,partopsep=0pt,parsep=\parskip,topsep=5pt}
\setdescription{itemsep=0pt,partopsep=0pt,parsep=\parskip,topsep=5pt}
%定理
\makeatletter
\thm@headfont{\sc}
\makeatother
\newtheorem{theorem}{Theorem}
%%%%%%%%%%%%% C++ Code
\usepackage{color}
\usepackage{xcolor}
\definecolor{keywordcolor}{rgb}{0.8,0.1,0.5}
\usepackage{listings}
\lstset{breaklines}%这条命令可以让LaTeX自动将长的代码行换行排版
\lstset{extendedchars=false}%这一条命令可以解决代码跨页时,章节标题,页眉等汉字不显示的问题
\lstset{ %用于设置语言为C++
    keywordstyle=\color{keywordcolor} \bfseries, 
    %identifierstyle=,
    basicstyle=\ttfamily, 
    commentstyle=\color{blue} \textit,
    stringstyle=\ttfamily, 
    showstringspaces=false,
    frame=shadowbox, %边框
    %captionpos=b
}


%\setCJKmainfont[BoldFont = 黑体]{宋体}
\setlength{\parindent}{2em}
%如果不要缩进 用\noindent
\title{Deep Learning for Natural Language Processing \\ (cs224d) \\ Lecture Notes}

\author{\large Yihong Gu\footnote{gyh15@mails.tsinghua.edu.cn}}

\date{}

\begin{document}

\maketitle

\tableofcontents

\newpage

\section{NLP and DL: introduction}

\subsection{Natural Language Processing}

\subsubsection*{Hierarchical Stages}

首先,需要了解关于Natural Language Processing的几个层次:

\begin{itemize}
	\item \textbf{Phonetic Analysis}/\textbf{OCR}阶段:主要解决的是自然语言的输入端的问题,其中\textbf{Phonetic}为语音、\textbf{OCR}为文本。
	\item \textbf{Morphological Analysis}: 主要解决词的合成的问题,比如词的前缀和后缀的意义(e.g. un-)。
	\item \textbf{Syntactic Analysis}: 主要解决句法合成的问题,研究词如何通过语法结构进行组合。
	\item \textbf{Semantic Interpretation}: 主要解决句子含义(meaning)的问题。
	\item \textbf{Discourse Processing}: 主要解决整篇文章的理解的问题。
\end{itemize}

\subsubsection*{Difficulty}

传统的NLP的困难之处主要在以下两个方面,第一个方面是知识(knowledge)的表示,以下面句子为例

\begin{center}
\begin{quote}
Jane hit June and then \textbf{she} [fell/ran]
\end{quote}
\end{center}

显然,fell和ran分别让she有了不同的指代(Jane/June),如果要让机器能够准确作出正确的指代关系的话,仅仅考虑单词的意思和语法结构是不够的,机器需要理解hit和fell/ran之间的逻辑关系,这需要进一步的先验知识。

第二个方面是模糊性(ambiguity),即多重含义。

\subsection{Deep Learning}

\subsubsection*{Representation Learning}

我们考虑一个传统的(预测类)机器学习过程,我们把它分成两部分

$$\text{Machine Learning} = \text{Feature Engineering}+\text{Learning Algorithm}$$

在这样的过程中,我们拿到数据(data),先要人工设计数据的特征表示(representation),然后再使用特定的机器学习算法。通常而言,特征表示在整个过程的地位更高并且需要花费更多的精力。

而Deep Learning是什么?我们可以把它看成一种特征学习(representation learning),它能够从大量的数据中自动地学习特征。在具体实践中,Deep Learning更加灵活,适用性也更强,一个适用于某个问题的DL模型可以直接运用于另外一个问题上。

\subsubsection*{Visualization}

值得注意的是,可视化(\textbf{visualization})在DL中非常重要,我们通常会(使用PCA等方法)把我们从DL中学到的feature投影到二维空间作可视化,我们可以从中发现一些非常有趣的性质:对于NLP问题来说,一个非常广泛的性质在于意思相近的单词/短语/句子在这样的欧几里得空间中也靠的更近。

\subsubsection*{DL+NLP}

我们考虑DL在NLP上的应用,其核心在于把单词/短语/句子映射成$d$维空间中的向量,进而进行进一步的分析。

\section{Word Vector}

Word Vector的意义在于把单词映射成$d$维空间中的向量。关于word vector(以及之后的NLP),我们有一些相关的术语:

\begin{itemize}
	\item \textbf{corpus}: 指用来作训练的文本全集。
	\item \textbf{token}: 文本中的某个单一的元素,可以是一个单词,也可以是一个标点,也可以是开始符/结束符。
\end{itemize}

传统的NLP中,关于单词有三种表示方法:

\begin{itemize}
	\item one-hot: 单词的词向量$\in \mathbb{R}^{\lvert V\rvert}$,其中$\lvert V\rvert$为单词总数。
	\item taxonomy: 建立词与词之间的从属关系(is-a),比如flower从属于plant(flower \textbf{is-a} plant),相关的工作有WordNet。
	\item synonym set: 建立同义词/近义词集合。
\end{itemize}

\subsection{Statistical Based Method: SVD}

\subsubsection*{Model}

首先,我们通过文本构造\textbf{Window based Co-occurence Matrix},考虑构造这样一个矩阵$X \in \mathbb{M}_{\lvert V\rvert \times \lvert V\rvert}$,其\textbf{window size}为$W$,那么,如果单词$w_j$在单词$w_i$的大小为$W$的window中出现,$X_{i,j}$就累加$1$,下面是一个例子

\begin{quote}
\begin{itemize}
	\item[1.] I enjoy flying.
	\item[2.] I like NLP.
	\item[3.] I like deep learning.
\end{itemize}
\end{quote}

其对应的co-occurence matrix $X$为:

\begin{equation}X=\label{cooccurence_matrix}\bordermatrix{
& I & like & enjoy & deep & learning & NLP & flying & .\cr
       I & 0 & 2 & 1 & 0 & 0 & 0 & 0 & 0\cr
    like & 2 & 0 & 0 & 1 & 0 & 1 & 0 & 0\cr
   enjoy & 1 & 0 & 0 & 0 & 0 & 0 & 1 & 0\cr
    deep & 0 & 1 & 0 & 0 & 1 & 0 & 0 & 0\cr
learning & 0 & 0 & 0 & 1 & 0 & 0 & 0 & 1\cr
     NLP & 0 & 1 & 0 & 0 & 0 & 0 & 0 & 1\cr
  flying & 0 & 0 & 1 & 0 & 0 & 0 & 0 & 1\cr
       . & 0 & 0 & 0 & 0 & 1 & 1 & 1 & 0
}\end{equation}

注意到这里的矩阵$X$为对称矩阵,我们对其作SVD,令$X=\mathcal{U}\mathrm{S}\mathcal{V}$,其中$\mathcal{U},\mathcal{V} \in \mathbb{M}_{\lvert V\rvert\times \lvert V\rvert}$并且是正定矩阵,$\mathrm{S} \in \mathbb{M}_{\lvert V\rvert\times\lvert V\rvert}$并且是对角矩阵,记$\sigma_i=S_{i,i}$,取$\mathcal{U}$的前$K$列$\mathcal{U}^{[,1:K]}$作为最后的$K$维word vector,这时候第$i$行的向量$\mathrm{u}_i=\mathcal{U}^{[,1:K]}_i$即为单词$w_i$的word vector。

\subsubsection*{Drawbacks and Extensions}

这样的模型主要有以下几个问题:

\begin{itemize}
	\item 由矩阵本身引起的问题:矩阵太大并且非常稀疏、加入新单词后拓展非常耗时间。
	\item 关于co-occurence的问题:需要解决部分词出现太频繁所导致的问题(比如the etc.)。
\end{itemize}

我们考虑这两类的问题的解决方案,首先是co-occurence的问题,我们可以用以下的方案来解决:

\begin{itemize}
	\item [1.] 令$X^*_{i,j}=min(X_{i,j}, t)$,其中$t \sim 100$
	\item [2.] 使用pearson correlation,同时让负值取0从而代替原来的简单的计数。
	\item [3.] 使用ramp window:给window的不同位置加权,一个比较自然的想法是距离更近的位置的权值更高。
\end{itemize}

我们无法很好的解决第一个问题,这就引入了下面的\ref{word2vec}的内容。

\subsection{Iteration Based Methods: word2vec}
\label{word2vec}

我们在本节的模型中都会运用到以下记号:

\begin{itemize}
	\item $w_i$为单词集(vocabulary)中的第$i$个单词
	\item $n$为单词空间的维数
	\item $\mathcal{V}\in\mathbb{R}^{n\times\lvert V\rvert}$为input word matrix,其中$\mathrm{v}_i$是$\mathcal{V}$的第$i$列,表示单词$w_i$的input vector
	\item $\mathcal{U}\in\mathbb{R}^{n\times\lvert V\rvert}$output word matrix,其中$\mathrm{u}_i$是$\mathcal{U}$的第$i$列,表示单词$w_i$output vector
	\item window size为$m$。
\end{itemize}

\subsubsection*{CBOW Model}

在CBOW(Continuous Bag of Word)模型中,我们考虑一个问题:使用周围的单词(context)预测中间的单词(center word),我们考虑使用以下的预测模型结构:

假设我们需要预测的单词是$w^{(c)}$,其周围的单词为$w^{(c-m)},\cdots,w^{(c-1)},w^{(c+1)},\cdots,w^{(c+m)}$,为了方便表示,我们使用这些单词的one-hot表示形式$\mathrm{x}^{(c-m)},\cdots,\mathrm{x}^{(c-1)},\mathrm{x}^{(c+1)},\cdots,\mathrm{x}^{(c+m)}$,则相应的单词$w^{(c+i)}$所对应的input vector为

\begin{eqnarray}
\mathrm{v}^{(c+i)}=\mathcal{V} \mathrm{x}^{(c+i)}
\end{eqnarray}

我们对这$2m$个向量取平均,得到一个$n$维向量$\hat{\mathrm{v}}$

\begin{eqnarray}
\hat{\mathrm{v}}=\frac{1}{2m}(\mathrm{v}^{(c-m)}+\cdots+\mathrm{v}^{(c-1)}+\mathrm{v}^{(c+1)}+\cdots+\mathrm{v}^{(c+m)})
\end{eqnarray}

接下来我们得到了一个得分向量(score vector)$\mathrm{z}$,这里$\mathrm{z}\in\mathbb{R}^{\lvert V\rvert}$其第$i$个元素表示$\hat{\mathrm{v}}$和第$i$个单词的输出向量的相似程度,这里的相似程度用点乘的大小来度量):

\begin{eqnarray}
\mathrm{z}=\mathcal{U}\hat{\mathrm{v}}
\end{eqnarray}

最后我们把得分向量作一个softmax变换得到中心的单词的概率分布

\begin{eqnarray}
\hat{\mathrm{y}} = \mathrm{softmax}(\mathrm{z})
\end{eqnarray}

我们定义损失函数(统计/决策论意义上的)$H$为:

\begin{eqnarray}
H(\mathrm{y},\hat{\mathrm{y}})=-\sum_{y=1}^{\lvert V\rvert}{y_i\log(\hat{y}_i)}
\end{eqnarray}

实际上,如果让$\mathrm{y}$为center word的one-hot向量,其中$y_k=1$,那么,$H(\mathrm{y},\hat{\mathrm{y}})$可以简化为

\begin{eqnarray}
H(\mathrm{y},\hat{\mathrm{y}})=-y_k\log(\hat{y}_k)
\end{eqnarray}

我们写出最后的object function的化简形式

\begin{eqnarray}
J &=& -\log\mathbb{P}(\mathrm{u_c}\lvert w^{(c-m)},\cdots,w^{(c-1)},w^{(c+1)},\cdots,w^{(c+m)}) \\
  &=& -\log\frac{\exp(\mathrm{u}_c^T\hat{\mathrm{v}})}{\sum_{j=1}^{\lvert V\rvert}{\exp(\mathrm{u}_j^T\hat{\mathrm{v}})}} \\
  &=& -\mathrm{u}_c^T\hat{\mathrm{v}} + \log\sum_{j=1}^{\lvert V\rvert}{\exp(\mathrm{u}_j^T\hat{\mathrm{v}})}
\end{eqnarray}

其中$\mathrm{u}_c$为center word的output vector。

结合上面的表达式,我们需要对这两个vector有一个直观的认识,实际上,我们的优化目标是让每个center word周围的单词的input vector的平均值尽量接近其output vector。

\subsubsection*{Skip-Gram Model}

Skip-Gram Model的方向正好和CBOW相反,在Skip-Gram中,我们给定中间的center word,需要预测周围的context,为了简化描述,我们直接给出概率形式的表达:式[\ref{skipprobsingle}]给出了给定中间单词$w_c$,单词$w_o$在其window中出现的概率;式[\ref{skiplossprob}]给出了损失函数的形式,这里我们使用了近一步的独立性假设$\mathbb{P}(w_{c-m},\cdots,w_{c-1},c_{c+1},\cdots,w_{c+m}\lvert w_c)=\prod_{j=-m,j\neq 0}^{m} \mathbb{P}(w_{c+j}\lvert w_c)$(虽然这非常不符合实际);我们把[\ref{skipprobsingle}]代入[\ref{skiplossprob}]并且进一步化简得到式[\ref{skiploss}]

\begin{eqnarray} \label{skipprobsingle}
\mathbb{P}(w_o\lvert w_c) &=& \frac{\exp(\mathrm{u}_o^T\mathrm{v}_c)}{\sum_{k=1}^{\lvert V\rvert}{\exp(\mathrm{u}_k^T\mathrm{v}_c)}} \\
\label{skiplossprob}
J &=& -\log\prod_{j=-m,j\neq 0}^{m} \mathbb{P}(w_{c+j}\lvert w_c) \\
\label{skiploss}
J &=& -\sum_{j=-m,j\neq 0}^{m}{\mathrm{u}_{c+j}^T\mathrm{v}_c}+2m\log\sum_{k=1}^{\lvert V\rvert}{\exp(\mathrm{u}_k^T\mathrm{v}_c)}
\end{eqnarray}

我们给出梯度的表达式:

\begin{eqnarray}
\frac{\partial J}{\partial \mathrm{v}_c}&=&-\sum_{j=-m,j\neq 0}^{m}{\mathrm{u}_{c+j}^T}+2m\sum_{x=1}^{\lvert V\rvert}{\frac{\exp(\mathrm{u}_x^T\mathrm{v}_c)}{\sum_{k=1}^{\lvert V\rvert}{\exp(\mathrm{u}_k^T\mathrm{v}_c)}}\mathrm{u}_x^T}\\
\frac{\partial J}{\partial \mathrm{u}_x}&=&m \frac{\exp(\mathrm{u}_x^T\mathrm{v}_c)}{\sum_{k=1}^{\lvert V\rvert}{\exp(\mathrm{u}_k^T\mathrm{v}_c)}}\mathrm{v}_c^T - y_x\mathrm{v}_c^T
\end{eqnarray}

但是,最后我们的形式仅考虑一对$\mathbb{P}(o\lvert c)$,最后的object function的结果如式[\ref{skipof}],梯度的表达式如式[\ref{skipgrad1}]和[\ref{skipgrad2}]

\begin{eqnarray}
\label{skipof}
J &=& -\mathrm{u}_{c+j}^T\mathrm{v}_c + \log\sum_{k=1}^{\lvert V\rvert}{\exp(\mathrm{u}_k^T\mathrm{v}_c)} \\
\label{skipgrad1}
\frac{\partial J}{\partial \mathrm{v}_c}&=&-\mathrm{u}_{c+j}^T+\sum_{x=1}^{\lvert V\rvert}{\frac{\exp(\mathrm{u}_x^T\mathrm{v}_c)}{\sum_{k=1}^{\lvert V\rvert}{\exp(\mathrm{u}_k^T\mathrm{v}_c)}}\mathrm{u}_x^T} \\
\label{skipgrad2}
\frac{\partial J}{\partial \mathrm{u}_x}&=&\frac{\exp(\mathrm{u}_x^T\mathrm{v}_c)}{\sum_{k=1}^{\lvert V\rvert}{\exp(\mathrm{u}_k^T\mathrm{v}_c)}}\mathrm{v}_c^T - y_x\mathrm{v}_c^T
\end{eqnarray}

\subsubsection*{Negative Sampling}

考虑到以上的模型在实际计算中运算量非常大,所以我们考虑另外一种近似方法,这种近似方法的阐述也分成两个部分:第一个部分是object function,第二个部分是gradient。

首先,我们考虑Skip-Gram Model的Negative Sampling的近似优化方法:

我们使用$(x,c)$来分别表示$(\text{context},\text{center})$,同时记$D=\{(x,c)\lvert w_x=w^{(c+j)}, -m\le j\le m, j\neq 0\}$,那么,我们可以得到另外一种$\mathcal{P}(w_o\lvert w_c)$的表达式[\ref{nsskipprobsingle}]

\begin{eqnarray}
\label{nsskipprobsingle}
\mathbb{P}\big((o,c)\in D\big) = \sigma(\mathrm{u}_o^T\mathrm{v}_c) = \frac{1}{1+\exp(-\mathrm{u}_o^T\mathrm{v}_c)}
\end{eqnarray}

同时使用独立性条件,我们写出object function的表达式[\ref{ofskip}],将[\ref{nsskipprobsingle}]的结果代入,得到最终的表达式[\ref{ofskipf}]

\begin{eqnarray}
\label{ofskip}
J &=& -\log\Big\{\prod_{(o,c) \in D}{\mathbb{P}\big((o,c) \in D\big)}\prod_{(x,c)}{\mathbb{P}\big((w,c) \in \bar{D}\big)}\Big\} \\
\label{ofskipf}
J &=& -\sum_{(o,c) \in D}{\log(\sigma(\mathrm{u}_o^T\mathrm{v}_c))} - \sum_{(w,c) \in \bar{D}}{\log(\sigma(-\mathrm{u}_w^T\mathrm{v}_c))}
\end{eqnarray}

我们把$\bar{D}$称为nagative corpus,在具体求解的时候,我们在$\bar{D}$中随机抽$K$个组成(一对的)object function [\ref{ofnsskip}],其梯度为 [\ref{ofnsskipgrad1}], [\ref{ofnsskipgrad2}] 和 [\ref{ofnsskipgrad3}],其中每个单词被抽到的概率服从分布 [\ref{ofnsdis}]:

\begin{eqnarray}
\label{ofnsskip}
J &=& -\log(\sigma(\mathrm{u}_{c+j}^T\mathrm{v}_c)) - \sum_{k=1}^K{\log(\sigma(-\bar{\mathrm{u}}_k^T\mathrm{v}_c))}\\
\label{ofnsskipgrad1}
\frac{\partial J}{\partial \mathrm{v}_c} &=& -\big(1-\sigma(\mathrm{u}_{c+j}^T\mathrm{v}_c)\big)\mathrm{u}_{c+j}^T + \sum_{k=1}^K{\big(1-\sigma(-\bar{\mathrm{u}}_k^T\mathrm{v}_c)\big)\bar{\mathrm{u}}_k^T} \\
\label{ofnsskipgrad2}
\frac{\partial J}{\partial \mathrm{u}_{c+j}} &=& -\big(1-\sigma(\mathrm{u}_{c+j}^T\mathrm{v}_c)\big)\mathrm{v}_{c}^T \\
\label{ofnsskipgrad3}
\frac{\partial J}{\partial \bar{\mathrm{u}}_k} &=& \big(1-\sigma(-\bar{\mathrm{u}}_k^T\mathrm{v}_c)\big)\mathrm{v}_{c}^T \\
\label{ofnsdis}
\mathbb{P}(w) &\propto& \big(\text{frequency of word $w$}\big)^{3/4}
\end{eqnarray}

同样,我们考虑CBOW Model的Negative Sampling的近似优化方法:

沿用之前的记号,类似的,我们可以得出object function:

\begin{eqnarray}
J &=& -\log(\sigma(\mathrm{u}_c^T\hat{\mathrm{v}})) - \sum_{w \neq c}{\log(\sigma(-\mathrm{u}_w^T\hat{\mathrm{v}}))}
\end{eqnarray}

同样,我们可以得出(一对的)object function [\ref{ofnscbow}],梯度 [\ref{ofnscbowgrad1}], [\ref{ofnscbowgrad2}] 和 [\ref{ofnscbowgrad3}]。

\begin{eqnarray}
\label{ofnscbow}
J &=& -\log(\sigma(\mathrm{u}_c^T\hat{\mathrm{v}})) - \sum_{k=1}^K{\log(\sigma(-\bar{\mathrm{u}}_k^T\hat{\mathrm{v}}))}\\
\label{ofnscbowgrad1}
\frac{\partial J}{\partial \mathrm{v}_{c+j}} &=& \frac{1}{2m}\Big\{-\big(1-\sigma(\mathrm{u}_{c}^T\hat{\mathrm{v}})\big)\mathrm{u}_{c}^T + \sum_{k=1}^K{\big(1-\sigma(-\bar{\mathrm{u}}_k^T\hat{\mathrm{v}})\big)\bar{\mathrm{u}}_k^T}\Big\} \\
\label{ofnscbowgrad2}
\frac{\partial J}{\partial \mathrm{u}_{c}} &=& -\big(1-\sigma(\mathrm{u}_{c}^T\hat{\mathrm{v}})\big)\hat{\mathrm{v}}^T \\
\label{ofnscbowgrad3}
\frac{\partial J}{\partial \bar{\mathrm{u}}_k} &=& \big(1-\sigma(-\bar{\mathrm{u}}_k^T\hat{\mathrm{v}})\big)\hat{\mathrm{v}}^T \\
\end{eqnarray}

\subsection{Evaluation: Intrinsic and Extrinsic}

\section{Neural Networks}

\subsection{Neuron, Terminology, Foward Propagation}

\subsection{Back Propagation, Computational Graph}

\subsection{Tips and Tricks}

\section{Recurrent Neural Networks}

\section{Recursive Neural Networks}

\section{}
\end{document}